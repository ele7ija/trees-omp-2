\section{Увод}
У овом раду, анализиран је проблем проналажења суме у бинарном стаблу. Проблем можемо представити и питањем: "Да ли чворови бинарног стабла којима је придружен
природан број у било којој путањи од корена до листа дају задати збир?".

Најчешће решење овог има асимптотску временску сложеност извршавања $O(n)$ gde $n$ представља величину улаза tj. број чворова у графу.

У овом раду, предложено је паралелно решење проблема које узима у обзир независност обраде деце одређеног чвора.

Постигнута асимптотска сложеност паралелног решења је приближно: $O(n/p)$, где је $n$ величина улаза тј. број чворова у графу, а $p$ број јединица паралелног извршавања.

Рад је конципиран на следећи начин: прво ће сам проблем бити описан као и његово секвенцијално решење, потом ће бити приказано паралелно решење и
, на крају, биће представљени подаци о ефикасности секвенцијалног и паралелног решења.
\pagebreak