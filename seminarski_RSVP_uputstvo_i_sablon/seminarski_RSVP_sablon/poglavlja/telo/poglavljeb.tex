\section{Паралелно решење проналаска суме у бинарном стаблу}

\subsection{Опис паралелног решења}
Дефинишимо \textit{дете-стабло} као стабло чији је корен дете датог чвора.

Секвенцијално решење не користи чињеницу да се "лево" и "десно" дете-стабло могу независно обрађивати.
Међутим, у имплементационом смислу, оптимално решење није да се свако дете-стабло обрађује независно тј.
у посебном ОпенМП задатку, јер тиме се оптерећује извршно окружење ОпенМП-а тј. доста процесорског времена
се потроши на прекључивање контекста задатака уместо на ефективном раду и извршавању задатака.

Предлаже се паралелно решење које независно обрађује само стабла чији су корени на нивоу $l$, таквом да: $$l = max_i 2^i \leq \mathrm{omp\_get\_num\_threads()}$$.
Интуиција иза овога јесте да \textbf{паралелно} можемо да обрађујемо само онолико стабала колико имамо процесорских јединица на располагању.
Стога, треба да паралелно обрађујемо стабла са кореном на оном нивоу стабла на ком је број чворова такав да већ следећи ниво стабла има више чворова него
што програм има процесорске моћи.

У листингу \ref{code:paralelno} представљено је паралелно решење.

\begin{listing}
\inputminted{c}{kodovi/parallel.c}
\caption{Имплементација паралелног решења у језику \texttt{C}}
\label{code:paralelno}
\end{listing}

\subsection{Сложеност паралелног решења}

Асимптотска сложеност овог решења је приближно: $O(n/p)$, где је $n$ величина улаза тј. број чворова у графу, а $p$ број јединица паралелног извршавања.

\pagebreak