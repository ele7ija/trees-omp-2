\section{Закључак}

У овом раду, анализиран је проблем проналажења суме у бинарном стаблу.
Дат је приказ проблема и његовог секвенцијалног решења.
Потом је приказано и његово паралелно решење.
На крају, приказани су резултати који указују на линеарно убрзање доласка до решења са порастом броја процесуирајућих јединица за стабла са великим бројем чворова ($n > 10 000$).

Даљи рад на ову тему требало би узме у обзир да се са вредношћу $l$ дефинисану једначином \ref{eq:myequation} потенцијално не долази до најбољих резултата у општем случају.
То је због тога што се, са тако дефинисаном вредношћу креира укупно више задатака него што има процесуирајућих јединица.
Ово доводи до тога да, понекад, чвор на нивоу $l-1$ ни не крене да се извршава док неки од чворова нивоа $l$ не заврше своју обраду.
За неке вредности ПЈ, потенцијално би било ефикасније да се дефинише $l_{fixed}$ као $l_{fixed} = l - 1$.
Са вредношћу $l_{fixed}$ сигурни смо да сви генерисани задаци имају увек барем једну слободну процесуирајући јединицу.